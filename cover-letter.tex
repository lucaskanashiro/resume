%%%%%%%%%%%%%%%%%%%%%%%%%%%%%%%%%%%%%%%%%
% Friggeri Cover Letter
% XeLaTeX Template
% Version 1.0 (1/11/15) 
%
% This template has been downloaded from:
% https://github.com/mlda065/friggeri-letter
%
% Original author:
% Matthew Davis, based on code by 
% Adrien Friggeri (adrien@friggeri.net)
% https://github.com/afriggeri/CV
%
% License:
% CC BY-NC-SA 3.0 (http://creativecommons.org/licenses/by-nc-sa/3.0/)
% 
% Important notes:
% This template needs to be compiled with XeLaTeX
% You may need to compile twice for the header to appear.
%
%%%%%%%%%%%%%%%%%%%%%%%%%%%%%%%%%%%%%%%%%
  
\documentclass[a4paper,english]{friggeri-letter}

\usepackage{babel}

\begin{document}

\header{Lucas}{\ Kanashiro}{Software Engineer} % Your name and current job title/field

\address{ 
SHVP Rua 8 chácara 195 casa 4\\
Brasília, DF, 72006-810\\
Brazil
}


\letter{ 
   Canonical Ltd. \\ 
} 


   
  

\opening{Dear,} 

%Explain why you want the job, and summarise why you deserve the job.

I am applying for "QA Engineer - Server Enablement" position, I reached this
vacancy in the jobs section of your website. I am a software engineer and now
I am finishing my master degree in computer science. I am also based in Brazil.
In academia I did some research on FLOSS topic and since that I am passionate
about free software world. I am a former Debian GSoC student and a Debian
developer for 2 years, and one of my carrers goal is to continue working with
free software and Canonical is the right place to accomplish it.

Since my undergraduate I have done quality assurance work and noticed how it is
important in order to deliver value to the customers. In Debian, I did some
work during my first GSoC related to QA too, fixing some test suites.
Moreover, I am contributing with debci (Debian continuous integration system)
recently, which runs autopkgtest in background testing the installed packages
in a Debian system. Furthermore, handle large-scale workloads in highly
distributed systems is what I am working on my master, which could help to
test this kind of requirement. About network technologies, I am not an expert
since this was not my focus until now, but I already helped to manage local
networks in research labs, thus I know a bit about it but need to improve my
skills in this area for sure. Because my contributions to Debian I have a good
knowlodge about Python programming and Shell scripting. The motivation to work
on Canonical with some very skilled and maybe known guys will lead me to give
my best and I am totally able and open to learn whatever is needed and fill the
gaps.

I would appreciate the opportunity to join the team as a QA engineer. This is
my first option always, do something that really impacts people around the
world in a company that was born and breathed free software.


\closing{
   Yours Sincerely\\
   Lucas Kanashiro} 

\end{document}
