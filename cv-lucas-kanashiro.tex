%%%%%%%%%%%%%%%%%%%%%%%%%%%%%%%%%%%%%%%%%
% Friggeri Resume/CV
% XeLaTeX Template
% Version 1.2 (3/5/15)
%
% This template has been downloaded from:
% http://www.LaTeXTemplates.com
%
% Original author:
% Adrien Friggeri (adrien@friggeri.net)
% https://github.com/afriggeri/CV
%
% License:
% CC BY-NC-SA 3.0 (http://creativecommons.org/licenses/by-nc-sa/3.0/)
%
% Important notes:
% This template needs to be compiled with XeLaTeX and the bibliography, if used,
% needs to be compiled with biber rather than bibtex.
%
%%%%%%%%%%%%%%%%%%%%%%%%%%%%%%%%%%%%%%%%%

% Add 'print' as an option into the square bracket to remove colors from this template for printing
\ifdefined\nocolors
  \documentclass[print]{friggeri-cv}
\else
  \documentclass[]{friggeri-cv}
\fi

\addbibresource{bibliography.bib} % Specify the bibliography file to include publications

\begin{document}

\header{Lucas}{\ Kanashiro}{Software Engineer} % Your name and current job title/field

%----------------------------------------------------------------------------------------
%	SIDEBAR SECTION
%----------------------------------------------------------------------------------------

\begin{aside} % In the aside, each new line forces a line break
\section{contact}
SHVP Rua 8 chácara 195 casa 4
Brasília, DF 72006-810
Brazil
~
+55(61) 98500-4565
~
\href{mailto:kanashiro@riseup.net}{kanashiro AT riseup}
\href{http://blog.kanashiro.xyz}{blog.kanashiro.xyz}
\href{https://github.com/lucaskanashiro}{github}
\href{https://salsa.debian.org/kanashiro}{Debian salsa}
\href{https://www.linkedin.com/in/lucas-kanashiro-076821ba/}{LinkedIn}
\href{https://twitter.com/lucas\_kanashiro}{Twitter}
\section{languages}
Portuguese - native
English - advanced
\section{programming}
Python, Ruby, Perl, Shell script 
\section{software}
GNU/Linux, Ansible, Chef, SaltStack, Docker, Vagrant, git
\end{aside}

%----------------------------------------------------------------------------------------
%	EDUCATION SECTION
%----------------------------------------------------------------------------------------

\section{education}

\begin{entrylist}

%------------------------------------------------

\entry
{2016--2019}
{Master's degree -- {\normalfont Computer Science}}
{University of São Paulo, Brazil}
  {\emph{Simulation-based Experimentation for Smart City applications} \\ Developed large scale simulations of Smart Cities environment, and used these simulations to perform tests and experiments on Smart Cities platforms. The simulator has been developed in Erlang, taking advantage of the native actors' implementation.}

%------------------------------------------------

\entry
{2010--2015}
{Bachelor -- {\normalfont Software Engineer}}
{University of Brasília, Brazil}
  {\emph{A study on low complexity models to predict flaws in the Linux source code} \\ Defined statistical models that tries to fit a distribution of number of weaknesses in source code of Linux. We analyzed almost 400 versions of Linux, including alpha and beta releases. I received a honorable mention related to my performance during the undergraduation. }

%------------------------------------------------

\end{entrylist}

%----------------------------------------------------------------------------------------
%	WORK EXPERIENCE SECTION
%----------------------------------------------------------------------------------------

\section{experience}

\subsection{Work}

\begin{entrylist}

%------------------------------------------------

\entry
{2018--current}
{Collabora Ltd}
{Remote}
{\emph{Software Engineer} \\
  Maintaining softwares in Linux and BSD distributions, building infrastructure
  to build Linux distriubutions, creating Debian derivatives, customizing Linux
  images to run on cloud providers. Contributing to many open source projects.
}

%------------------------------------------------

\entry
{2017--current}
{Freexian SARL}
{Remote}
{\emph{Debian Long Term Support Consultant} \\
  Fix security issues in Debian Long Term Support (LTS) version.
}

%------------------------------------------------

\entry
{2014--2016}
{LAPPIS}
{University of Brasília, Brazil}
{\emph{Software Developer} \\
  Worked as a software developer on Portal do Software Público
  Brasileiro - a platform for hosting and development of Brazilian government
  FLOSS. \\
Detailed achievements:
\begin{itemize}
  \item Worked with the DevOps team on a Continuous Delivery strategy
  \item Performed deploys in production environment
  \item Developed features and bug fixes for Python and Ruby applications
\end{itemize}
  }

%------------------------------------------------

\entry
{2015--2015}
{Google Summer of Code - Debian Project}
{Debile}
{\emph{Software Engineer Intern} \\
  Worked on Debile project, a smart build package system that make static
  analysis on source code of Debian packages. \\
Detailed achievements:
\begin{itemize}
  \item Added new checkers in Debile
  \item Improve code coverage
  \item Deployed the latest version
\end{itemize}}

%------------------------------------------------

\entry
{2014--2014}
{Google Summer of Code - Debian Project}
{Debci}
{\emph{Software Engineer Intern} \\
  Worked on Debci project, the continuous integration tool used by Debian in
  its packages. \\
Detailed achievements:
\begin{itemize}
  \item Sent multiple patches for Debian fixing autopkgtest failures
\end{itemize}
}

%------------------------------------------------

\end{entrylist}

\subsection{Free and Open Source Projects}

\begin{entrylist}

\entry
{2014--Now}
{Debian Project}
{Remote}
{\emph{Debian Developer} \\
Detailed achievements:
  \begin{itemize}
    \item EFI team: testing Secure Boot
    \item Cloud team: maintaining Google cloud agent package and adapting
    images to run on GCE
    \item Perl Team member: packaging and maintaining software
    \item Ruby Team member: packaging and maintaining software
    \item Help organizing Debian related events and meetings in Brazil
    \item QA Team contributor: sent patches for distro-tracker, lintian and
    debci projects 
    \item Google Summer of Code mentor: help newcommers to contribute with
    distro-tracker project
    \item Sent some patches to upstream projects
  \end{itemize}}

\entry
{2018--Now}
{FreeBSD Project}
{Remote}
{\emph{Ports contributor} \\
Detailed achievements:
  \begin{itemize}
    \item Maintaining Google Cloud agent packages in FreeBSD ports tree
  \end{itemize}}
%------------------------------------------------


\end{entrylist}

\subsection{Scholarships}

\begin{entrylist}

\entry
{2013--2014}
{CQTS - Centro de Qualidade e Testes de Software}
{University of Brasília, Brazil}
{\emph{Software Developer} \\
  Worked on testing automation of Android apps user interfaces, in the
  context of a partnership between university and Positivo (Brazilian company) \\
Detailed achievements:
\begin{itemize}
  \item Developed an automated test suite for Android apps interface
\end{itemize}}

%------------------------------------------------
\end{entrylist}
%----------------------------------------------------------------------------------------
%	COMMUNICATION SKILLS SECTION
%----------------------------------------------------------------------------------------

\section{Events participation}

\begin{entrylist}

%------------------------------------------------

\entry
{2018}
{DebConf 18 - Debian Conference}
{Hsinchu, Taiwan}
{Discuss many contributions and future works with others developers. Moreover,
helped organizing the conference since Brazil will host DebConf 19 and I am
part of the organization team}

\entry
{2017}
{II Seminário "Software e Cultura no Brasil"}
{São Bernardo do Campo, SP, Brazil}
{Introduced the Debian Project community and described the distribution release process.}

\entry
{2017}
{DebConf 17 - Debian Conference}
{Montreal, Canada}
{Discuss many contributions and future works with others developers.}

\entry
{2017}
{MiniDebConf Curitiba}
{Curitiba, PR, Brazil}
{Organized a Bug Squashing Party before Debian Stretch release.}

\entry
{2017}
{Debian Release Party}
{São Paulo, SP, Brazil}
{Organized the release party and disseminate Debian to the audience.}

%------------------------------------------------

\entry
{2016}
{DebConf 16 - Debian Conference}
{Cape Town, South Africa}
{First DebConf as a Debian developer.}

\entry
{2016}
{MiniDebConf Curitiba}
{Curitiba, PR, Brazil}
{Hands-on work and a packaging tutorial.}

\entry
{2016}
{FISL - Fórum Internacional de Software Livre}
{Porto Alegre, RS, Brazil}
{Presented some talks about Debian and Web of Trust.}

%------------------------------------------------

\entry
{2015}
{DebConf 15 - Debian Conference}
{Heidelberg, Germany}
{As a Google Summer of Code student, presented a talk about my internship.}

\entry
{2015}
{FISL - Fórum Internacional de Software Livre}
{Porto Alegre, RS, Brazil}
{Presented a talk about Static Analysis.}

%------------------------------------------------

\entry
{2014}
{DebConf 14 - Debian Conference}
{Portland, USA}
{As a Google Summer of Code student I was presented to the community.}

\entry
{2014}
{FISL - Fórum Internacional de Software Livre}
{Porto Alegre, RS, Brazil}
{Presented a talk about the development status of Portal do Software Público Brasileiro.}


\end{entrylist}

%----------------------------------------------------------------------------------------
%	PARTICIPATION IN EVENTS
%----------------------------------------------------------------------------------------

\section{Extracurricular activities}

\begin{entrylist}

%------------------------------------------------

\entry
{2017}
{Hackathon USP 2017}
{São Paulo, SP, Brazil}
  {Our team won an honorable mention with \href{https://devpost.com/software/collab-research}{CollabResearch}}

\entry
{2017}
{Hackathon SEMISH 2017}
{São Paulo, SP, Brazil}
  {Volunteer -- Assisted participants with technical advice}

\entry
{2013}
{CBSoft 2013 - Congresso Brasileiro de Software}
{Brasília, DF, Brazil}
  {Volunteer -- Assisted speakers and staff}

\end{entrylist}

%----------------------------------------------------------------------------------------
%	INTERESTS SECTION
%----------------------------------------------------------------------------------------

\section{interests}

\textbf{professional:} Free Software, Static Analysis, Computer Security, DevOps, Software Design, Programming, Release Engineering\\
\textbf{personal:} Soccer, Movies, Running, Travelling

%----------------------------------------------------------------------------------------
%	PUBLICATIONS SECTION
%----------------------------------------------------------------------------------------

\section{publications}

\printbibsection{article}{article in peer-reviewed journal} % Print all articles from the bibliography

\printbibsection{book}{books} % Print all books from the bibliography

\begin{refsection} % This is a custom heading for those references marked as "inproceedings" but not containing "keyword=brazil"
\nocite{*}
\printbibliography[sorting=chronological, type=inproceedings, title={international peer-reviewed conferences/proceedings}, notkeyword={brazil}, heading=bibheading]
\end{refsection}

\begin{refsection} % This is a custom heading for those references marked as "inproceedings" and containing "keyword=brazil"
\nocite{*}
\printbibliography[sorting=chronological, type=inproceedings, title={local peer-reviewed conferences/proceedings}, keyword={brazil}, heading=bibheading]
\end{refsection}

\printbibsection{misc}{other publications} % Print all miscellaneous entries from the bibliography

\printbibsection{report}{research reports} % Print all research reports from the bibliography

%----------------------------------------------------------------------------------------

\end{document}
